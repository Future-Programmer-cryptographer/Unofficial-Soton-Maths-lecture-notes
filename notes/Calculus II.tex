\title{MATH1065: Calculus II}
\author{Shub Das \\ sd2g25@soton.ac.uk}
\date{\today}

\maketitle

\newpage


\tableofcontents 

\part{Ordinary Differential Equations}

\chapter{First-Order Ordinary Differential Equations}

When dealing with ODEs \footnote{Get used to seeing them because they will appear everywhere}, there are a few main ways of solving them.  

I won't be including examples in most of the sections, mostly because well, just do a bunch of questions to get better at ODEs. 

\section{Separation of Variables}

This is when the ODE is in terms of $x$ and $y$ so that we can separate the variables 

$$
g(y) \frac{dy}{dx} = h(x) \implies \int g(y) \ dy = \int h(x) \ dx + c 
$$

\section{Integrating Factor}

This method is used to solve Linear 1st order ODEs. Say we have an ODE in the form of: 
$$
\frac{dy}{dx} + p(x) y = q(x) 
$$

Then we can introduce this term called the \textbf{integrating factor} such that: 
$$
\frac{d}{dx} [ y(x) e^{P(x})] = q(x) e^{P(x)} \text{ with } P(x) = \int p(x)\ dx 
$$
Then we multiply by the integrating factor (think back to converting it to a direct integration form). 
\\ Note that the constant appears after we multiply by the integrating factor, not before. 
$$ 
y(x) = e^{-P(x)} \left[ \int q(x) e^{P(x)} \ dx + c] \right]
$$



%Always wondered why the auxilary solution was what it is.. 


%What does it mean for the derivative to be a linear operator?? 


\part{Multivariable Calculus}


