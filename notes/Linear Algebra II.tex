\title{MATH1049: Linear Algebra II }
\author{Shub Das \\ sd2g25@soton.ac.uk}
\date{\today}

\maketitle

\newpage


\tableofcontents 

\chapter{Proofs}

\section{Misconceptions}

This chapter was not something which was covered explicitly in lectures. But I still feel that it is important to learn how to write proofs, in fact, how to write and understand proofs. 

With my experience of linear algebra so far, I found that most confusion in proofs comes from forgetting what objects we're working with. 

Eg: 
\begin{itemize}
    \item Saying \textit{$U$ is a subspace} means checking three specific properties 
    \item Assuming \textit{$x \in Col(A)$} means $x = A u$ for some vector $u$ 
\end{itemize}

The point I'm trying to make is that most proofs begins by assuming the hypotheses and nothing else. So we must avoid assuming what we're trying to prove. If the goal is to show that a set is a subspace, then we cannot assume closure until we have demonstrated it. 


\chapter{Groups}

Before I even define what a group is, \textbf{\href{https://www.southampton.ac.uk/courses/2026-27/modules/math2003}{MATH2003 Group Theory}} is a second year module which goes into more detail about group theory. 

If and when I write `this will be useful later', I will try and briefly mention why. 

% what this is 
% Why is this relevant to linear algebra 
% what on earth are we trying to do..? 

\section{Definitions}

\chapter{Fields and Vector Spaces}

\chapter{Bases}

\chapter{Linear Transformations}

\chapter{Determinants}

\chapter{Diagonalisability}


