\title{MATH1062: Computational Mathematics}
\author{Shub Das \\ sd2g25@soton.ac.uk}
\date{\today}

\maketitle



\tableofcontents 

\chapter{Linear Programming and Simplex}

\section{Purpose}


At its heart, Linear programming (LP) is all about min-maxing, with one restriction, namely linearity. So given limited resources and rules, what is the best thing we can do, and `best'  here means often maximising profit, or minimising cost. 

Most common mistakes in LP often come from modelling the wrong thing. Formulating an LP means translating a given situation into: 
\begin{itemize}
    \item An objection 
    \item Decision variables 
    \item Constraints 
\end{itemize}

\section{Objective functions and constraints}

Often what I find with computational heavy subjects is increasing boredom, so in my notes I will attempt to keep some curiosity alive

I will not be going through how to formulate, because that process is very algorithmic and notes do a far better job with their memorable examples of diet. If time wasn't a difficult constraint \footnote{Pun intended, I'd be tempted to come up with my own examples, but for the time being, enjoy some \textbf{\href{https://www.matoles.com/stardew-mip}{Stardew Valley fun}}}

\subsection{Decision variables}

These are the things we control- like how much to buy, allocate, transport, etc. 

\subsection{Objective function}

This is the thing we're trying to optimise or min/max. 


\subsection{Constraints}

These describe what is allowed, and they are usually inequalities like $Ax \leq b$

Geometrically speaking, each constraint takes off part of space, and the \textbf{feasible region} is what's left after applying all the constraints when optimising our objective function 

Always remember non-negativity constraints like $x \geq 0$. They exist because negative quantities often have no interpretation. 

\subsection{Feasible Region}

Graphically, this will be the set of all points $x$ that obey all the constraints 

\section{Link to Matrices}

LP is secretly linear algebra where optimisation is simply moving between bases and updating matrices. More on this will be updated... 



